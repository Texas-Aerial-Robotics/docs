A simple tool for labeling object bounding boxes in images, implemented with Python Tkinter.

\subsection*{Dependencies}


\begin{DoxyCode}{0}
\DoxyCodeLine{sudo apt-get install python-imaging-tk}
\end{DoxyCode}


\subsection*{Training instructions}

The following are a modified version of the instructions found on the yolo website


\begin{DoxyEnumerate}
\item Strat by collecting images. the goal is to collect close to 500 img/class
\item create a new folder in B\+Box-\/\+Label-\/\+Tool/\+Images
\item For multi-\/class task, modify \textquotesingle{}class.\+txt\textquotesingle{} with your own class-\/candidates and before labeling bbox,
\item run main.\+py and annotate images. The current tool requires that {\bfseries{the images to be labeled reside in /\+Images/001, /\+Images/002, etc. You will need to modify the code if you want to label images elsewhere}}. For multi-\/class choose the \textquotesingle{}Current Class\textquotesingle{} in the Combobox and make sure you click \textquotesingle{}Comfirm\+Class\textquotesingle{} button.
\item To create a new bounding box, left-\/click to select the first vertex. Moving the mouse to draw a rectangle, and left-\/click again to select the second vertex.
\begin{DoxyItemize}
\item To cancel the bounding box while drawing, just press {\ttfamily $<$Esc$>$}.
\item To delete a existing bounding box, select it from the listbox, and click {\ttfamily Delete}.
\item To delete all existing bounding boxes in the image, simply click {\ttfamily Clear\+All}.
\end{DoxyItemize}
\item After finishing one image, click {\ttfamily Next} to advance. Likewise, click {\ttfamily Prev} to reverse. Or, input an image id and click {\ttfamily Go} to navigate to the speficied image.
\begin{DoxyItemize}
\item Be sure to click {\ttfamily Next} after finishing a image, or the result won\textquotesingle{}t be saved.
\end{DoxyItemize}
\item run the convert.\+py script to format the annotations in yolo form
\item combine jour dataset\+\_\+list files using your favorite text editor into train.\+txt and test.\+txt. It is recommended to not train on one of your data sets to use as a accuracy metric.
\item Set up the darknet config.
\end{DoxyEnumerate}
\begin{DoxyItemize}
\item Create a file in darknet/cfg with the name $<$nameofdata$>$.data
\item Create a folder to store the output weights in. ex /home/eric/backup 
\begin{DoxyCode}{0}
\DoxyCodeLine{1 classes= 20}
\DoxyCodeLine{2 train  = <path-to-data>/train.txt}
\DoxyCodeLine{3 valid  = <path-to-data>/test.txt}
\DoxyCodeLine{4 names = data/voc.names}
\DoxyCodeLine{5 backup = <path-to-backup-folder>}
\end{DoxyCode}

\end{DoxyItemize}
\begin{DoxyEnumerate}
\item run {\ttfamily ./darknet detector train cfg/$<$nameofdata$>$.data cfg/yolov3.\+cfg $<$weights(if you want to continue training from a backup)$>$}
\end{DoxyEnumerate}

{\bfseries{Screenshot\+:}} 

\subsection*{Data Organization }

Label\+Tool

$\vert$

$\vert$--main.\+py $\ast$\# source code for the tool$\ast$

$\vert$

$\vert$--Images/ $\ast$\# direcotry containing the images to be labeled$\ast$

$\vert$

$\vert$--Labels/ $\ast$\# direcotry for the labeling results$\ast$

$\vert$ $\vert$--Labels-\/\+Yolo-\/\+Format $\ast$\# direcotry for the labeling results in yolo readable format$\ast$

$\vert$

$\vert$--Examples/ $\ast$\# direcotry for the example bboxes$\ast$ 