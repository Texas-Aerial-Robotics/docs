A simple tool for labeling object bounding boxes in images, implemented with Python Tkinter.

\subsection*{Dependencies}


\begin{DoxyCode}{0}
\DoxyCodeLine{sudo apt-get install python-imaging-tk}
\end{DoxyCode}
 \subsection*{Clone B\+Box-\/\+Label-\/\+Tool}


\begin{DoxyCode}{0}
\DoxyCodeLine{git clone https://github.com/Texas-Aerial-Robotics/BBox-Label-Tool.git}
\end{DoxyCode}


\subsection*{Training instructions}

The following are a modified version of the instructions found on the yolo website

{\itshape 1.} Start by collecting images. the goal is to collect close to 500 img/class

{\itshape 2.} create a new folder in B\+Box-\/\+Label-\/\+Tool/\+Images

{\itshape 3.} For multi-\/class task, modify \textquotesingle{}class.\+txt\textquotesingle{} with your own class-\/candidates and before labeling bbox,

ex 
\begin{DoxyCode}{0}
\DoxyCodeLine{person}
\DoxyCodeLine{dog}
\DoxyCodeLine{airplane}
\end{DoxyCode}


{\itshape 4.} run main.\+py and annotate images. The current tool requires that {\bfseries{the images to be labeled, reside in /\+Images/001, /\+Images/002, etc. You will need to modify the code if you want to label images elsewhere}}. For multi-\/class choose the \textquotesingle{}Current Class\textquotesingle{} in the Combobox and make sure you click \textquotesingle{}Comfirm\+Class\textquotesingle{} button.

{\itshape 5.} To create a new bounding box, left-\/click to select the first vertex. Moving the mouse to draw a rectangle, and left-\/click again to select the second vertex.
\begin{DoxyItemize}
\item To cancel the bounding box while drawing, just press {\ttfamily $<$Esc$>$}.
\item To delete a existing bounding box, select it from the listbox, and click {\ttfamily Delete}.
\item To delete all existing bounding boxes in the image, simply click {\ttfamily Clear\+All}.
\end{DoxyItemize}

{\itshape 6.} After finishing one image, click {\ttfamily Next} to advance. Likewise, click {\ttfamily Prev} to reverse. Or, input an image id and click {\ttfamily Go} to navigate to the speficied image.
\begin{DoxyItemize}
\item Be sure to click {\ttfamily Next} after finishing a image, or the result won\textquotesingle{}t be saved.
\end{DoxyItemize}

{\itshape 7.} We now need to convert the bbox labels into yolo readable form for this we will modify the convert script in {\ttfamily B\+Box-\/\+Label-\/\+Tool/\+Scripts}

the {\ttfamily mypath} variable should path to the labels folder for your data set. Yolo will look for the annotations in the same folder as the image, so path the {\ttfamily outpath} variable to the image folder. the dataset is the name of the data set ex {\ttfamily 002}. Then fill the classes array to match your {\ttfamily classes.\+txt}


\begin{DoxyCode}{0}
\DoxyCodeLine{mypath = "/home/eric/BBox-Label-Tool/Labels/002/" /}
\DoxyCodeLine{outpath = "/home/eric/BBox-Label-Tool/Images/002/"}
\DoxyCodeLine{dataSet = "002"}
\DoxyCodeLine{classes = ["person", "dog","airplane"]}
\end{DoxyCode}


{\itshape 8.} run the convert.\+py script to format the annotations in yolo form.

{\itshape 9.} combine your dataset\+\_\+list files using your favorite text editor into train.\+txt and test.\+txt. It is recommended to not train on one of your data sets to use as a accuracy metric.

{\itshape 10.} Set up the darknet .data file


\begin{DoxyItemize}
\item Create a file in darknet/cfg with the name $<$name-\/of-\/data$>$.data
\item Create a folder to store the output weights in. ex /home/eric/backup 
\begin{DoxyCode}{0}
\DoxyCodeLine{classes= 20}
\DoxyCodeLine{train  = <path-to-data>/train.txt}
\DoxyCodeLine{valid  = <path-to-data>/test.txt}
\DoxyCodeLine{names = data/<name-of-data>.names}
\DoxyCodeLine{backup = <path-to-backup-folder>}
\end{DoxyCode}

\end{DoxyItemize}

{\itshape 11.} Make your $<$name-\/of-\/date$>$.names file in {\ttfamily darknet/data}. This file should be a coppy of the classes.\+txt file in {\ttfamily B\+Box-\/\+Label-\/\+Tool}.

ex. 
\begin{DoxyCode}{0}
\DoxyCodeLine{person}
\DoxyCodeLine{dog}
\DoxyCodeLine{airplane}
\end{DoxyCode}


{\itshape 12.} Start your training run. You will fist need to select wich yolo model you would like to use the options are


\begin{DoxyItemize}
\item yolov1\+: Not recommended. this model is old
\item yolov2\+: more accurate, and faster.
\item yolov3\+: about as fast as v2, but more accurate. Yolo v3 has a high G\+PU ram requirement to train and run. If your graphics card does not have enough ram, use yolo v2
\item tiny-\/yolo\+: Very fast yolo model. Would recommend for application where speed is most important. Works very well on Nvidia Jetson
\end{DoxyItemize}

The below command start training with yolo v3. Change the .cfg file to train with a different model 
\begin{DoxyCode}{0}
\DoxyCodeLine{./darknet detector train cfg/<nameofdata>.data cfg/yolov3.cfg <weights(if you want to continue training from a backup)>}
\end{DoxyCode}


{\itshape 13.} After training has completed you can test the output weights by using the following command. 
\begin{DoxyCode}{0}
\DoxyCodeLine{./darknet detector test cfg/<nameofdata>.data cfg/yolov3.cfg <weights> <test-picture>}
\end{DoxyCode}


{\bfseries{Screenshot\+:}} 

\subsection*{Data Organization }

Label\+Tool

$\ast$$\ast$ $\vert$$\ast$$\ast$

$\ast$$\ast$$\vert$ --main.\+py source code for the tool$\ast$$\ast$

$\ast$$\ast$ $\vert$$\ast$$\ast$

$\ast$$\ast$$\vert$ --Images/ {\bfseries{direcotry containing the images to be labeled}} $\ast$$\ast$ $\vert$$\ast$$\ast$

$\ast$$\ast$ $\vert$ --Labels/ {\bfseries{direcotry for the labeling results}}

$\ast$$\ast$ $\vert$ $\ast$$\ast$ $\ast$$\ast$$\vert$ --Labels-\/\+Yolo-\/\+Format {\bfseries{direcotry for the labeling results in yolo readable format}} $\ast$$\ast$$\vert$$\ast$$\ast$

$\ast$$\ast$$\vert$$\ast$$\ast$ --Examples/ {\bfseries{direcotry for the example bboxes}} 